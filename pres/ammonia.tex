\documentclass[9pt]{beamer}

\usepackage[T1]{fontenc}
\usepackage{color}
\usepackage{graphicx}
\usepackage{natbib}
\usepackage{tikz}


\usetheme{Boadilla}

\usefonttheme{professionalfonts}

\title[Air Quality]{Air Quality - Agricultural Emissions}
\subtitle{}
\author{Max Callaghan}
\institute[MCC]{\includegraphics[height=1cm,width=2cm]{/home/max/Pictures/MCC_Logo_RZ_rgb.jpg}}

\newtheorem*{remark}{}

\bibliographystyle{apalike}

\begin{document}
	
\begin{frame}
	\titlepage
\end{frame}

\addtobeamertemplate{frametitle}{}{%
	\begin{tikzpicture}[remember picture,overlay]
	\node[anchor=north east,yshift=2pt] at (current page.north east) {\includegraphics[height=0.8cm]{/home/max/Pictures/MCC_Logo_RZ_rgb.jpg}};
	\end{tikzpicture}}

\begin{frame}{Agricultural Emissions}
	
	Outdoor air pollution contributes to millions of global premature deaths annually, with  most health impacts air linked to fine particulate matter (PM\textsubscript{2.5}) \citep{Lelieveld2015}. Nitrogen oxides (NO\textsubscript{x}) and Sulphir dioxide (SO\textsubscript{2}) are produced in the transport and power sectors and undergo further chemical reactions in the atmosphere to form PM\textsubscript{2.5}. Because these reactions are conditional on concentrations of ammonia (NH\textsubscript{3}), which is emitted in the agricultural sector, the optimal air pollution mitigation strategy should take into account all three sources, as well as primary carbonaceous aerosols of organic matter (OM) and black carbon (BC).
	
	Due to atmospheric reactions between alkaline ammonia (NH\textsubscript{3}) and acidic nitrogen oxides (NO\textsubscript{x}) and sulphur dioxide (SO\textsubscript{2}), that contribute to the formation of PM\textsubscript{2.5}, an optimal air pollution mitigation strategy should consider agricultural emissions (which primarily account for ammonia) as well as those from the transport and power sectors \citep{Wang2015, Lee2015}. Ammonia's contribution to premature mortality varies by region; in parts of Europe and North-Eastern America, reductions in Ammonia emissions may generate the largest reductions in mortality, although these results are sensitive to uncertainties about the relative toxicity of PM\textsubscript{2.5} particles by source \citep{Lee2015, Lelieveld2015}. Mitigation options for ammonia emissions are higher than other sectors, but behavioural changes towards reducing meat consumption would be highly effective, and would also generate health benefits for humans, a reduction in GHG emissions, a reduction in biodiversity loss, and the improvement of water quality \citep{Backes2016, Leip2015}.  

\end{frame}

\begin{frame}{Bibliography}
	\small
	\bibliography{/home/max/Documents/library/bibliography/library}
\end{frame}

\end{document}
